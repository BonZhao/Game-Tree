\begin{sloppypar}
 The Binary Decision tree model is perhaps the simplest model that computes Boolean functions: it charges only for reading an input variable. We proposed a theoretical binary game tree evaluation analysis process which could provide better lower bound compared with lower bound provided by previous analysis. Based on the concept of reluctant input, We study the power of randomness in this model with both directional and non-directional algorithm, and made the comparison of their efficiency with lower bound. These results are obtained via general and efficient methods for computing  lower bounds on the probabilistic complexity and based on a special type of uniform binary decision tree, which we call it NOR-tree, which is an equivalent transformation of AND/OR tree. As a result of study, we finally build up a more accurate analysis process. In order to test its versatility, we also apply this analysis process to two different types of binary game tree: Fibonacci tree and skew-F tree. Both of these two have more complicated structure for game tree evaluation. The results shows that this process could also be used for getting other type of game trees lower bound.

\newpage
\end{sloppypar}
%can you run bibliography on abstract?
%\bibliographystyle{plainnat}
%\bibliography{my_references}
